%		Helpfull Commands
%----------------------------------------------------------------
%	\\ 		This will make a new paragraph
%	\\*		New Line but not a paragraph
%	\linebreak		allow line to break here
%	\newline		request a new line
%	\newpage		request a new page
%	\pagebreak	encourage page break
%	\omega		This will print the omega char
%	_{exp}		This will do subscript text just replace exp



\documentclass[12pt]{amsart}
\usepackage{geometry} % see geometry.pdf on how to lay out the page. There's lots.
\usepackage{amsmath}
\geometry{letterpaper} % or letter or a5paper or ... etc
% \geometry{landscape} % rotated page geometry

% See the ``Article customise'' template for come common customisations

\title{CSCI 250 - Assignment 1}
\author{Peyton Grattino \& Alex Kunzler}
\date{\today} % delete this line to display the current date

%%% BEGIN DOCUMENT
\begin{document}
\maketitle
\tableofcontents

\addtocontents{toc}{\protect\setcounter{tocdepth}{1}}

\pagebreak
\setcounter{secnumdepth}{1}
\section*{Part A - 3 People Crossing}

\subsection*{a}
\begin{tabular}{ c c }
A is Alice & Painting is 20 \\
B is Bob & Vase is 12 \\
C is Carol & Sculpture is 8

\end{tabular}
\subsection*{b}
\begin{tabular}{ c c c }

$A_{20} B_{12} C_{8}$ & | | & No one \\
No one & | | & $A_{20} B_{12} C_{8}$

\end{tabular}

\subsection*{c}
The sum of the net worth of the individuals cannot be less than the sum of the value of the art on a given side of the river.

\subsection*{d}
\begin{tabular}{ c c c c }

$A_{20} B_{12}$ & | | & $C_{8}$ & \emph{C would cross the river with C's painting} \\

$A_{20} B_{12} C$ & | | & $_{8}$ & \emph{C would go to the starting side without their Sculpture} \\

$_{20} B C$ & | | & $A_{12}$ $_{8}$ & \emph{A would leave their Paining and cross the river with the Vase} \\

$A_{20} B C$ & | | & $_{12}$ $_{8}$ & \emph{A returns without the Vase and the Sculpture} \\

$A_{20}$ & | | & $B_{12} C_{8}$ & \emph{Both B and C head over to the other side of the river} \\

$A_{20} B_{12}$ & | | & $C_{8}$ & \emph{B heads back to the Western Bank with their Vase} \\

$B_{12}$ & | | & $A_{20} C_{8}$ & \emph{A crosses with their Painting} \\

$B_{12} C{8}$ & | | & $A_{20}$ & \emph{C would cross to the western bank their Sculpture} \\

$_{12}$ $_{8}$ & | | & $A_{20} B C$ & \emph{B and C cross with their art} \\

$A _{12}$ $_{8}$ & | | & $_{20} B C$ & \emph{A will cross to the west with no Painting} \\

$_{8}$ & | | & $A_{20} B_{12} C$ & \emph{A would bring over B's Vase} \\

$C_{8}$ & | | & $A_{20} B_{12}$ & \emph{C would return to get their art} \\

No one & | | & $A_{20} B_{12} C_{8}$ & \emph{C would return with their art}

\end{tabular}

\addtocontents{toc}{\protect\setcounter{tocdepth}{0}}
\section*{2 new people arrive}
\addtocontents{toc}{\protect\setcounter{tocdepth}{1}}

\subsection*{e}
Send Carol back because Elisha and Carol's net worth is equal to Dave's Statue. 

\subsection*{f}
A and B must leave with their artwork.

\newpage

\section*{Part B}
\subsection*{a}
Yes it is possible to AAA in to AAABC. The conversion is very simple:
\newline
AAA \quad -\textgreater \quad BC (Rule 3) %output AAA -> BC
\newline
BC \quad -\textgreater \quad AAAC (Rule 1) %output BC -> AAAC
\newline
AAAC \quad -\textgreater \quad AAAAAA (Rule 2) %output AAAC -> AAAAAA
\newline
AAAAAA \quad -\textgreater \quad AAABC (Rule 3) %output AAAAAA -> AAABC


\subsection*{b}
Using the rules 3, 1, and 2, in that order will add 3 As to any set of 3+ As. Using this knowledge going from eight As to 29 is just repeating rules 3, 1, and 2 7 times and you will have 29 As.

\subsection*{c}
Using the rules given the amount of As will only increment by 3. Due to this fact it would not be possible to get from 11 to 31.

\newpage

\section*{Part C}
\subsection*{a}
$\Omega$(g)
\subsection*{b}
O(g)
\subsection*{c}
$\Omega$(g)
\subsection*{d}
$\Theta$(g)
\subsection*{e}
O(g)
\subsection*{f}
$\Theta$(g)
\subsection*{g}
$\Theta$(g)
\subsection*{h}
$\Omega$(g)
\subsection*{i}
$\Theta$(g)
\subsection*{j}
O(g)
\subsection*{k}
$\Omega$(g)
\subsection*{l}
$\Theta$(g)
\subsection*{m}
$\Omega$(g)
\subsection*{n}
O(g)
\subsection*{o}
$\Theta$(g)
\subsection*{p}
$\Omega$(g)
\subsection*{q}
O(g)

\break

\subsection*{a}
Given $c < 1$ and that $c$ is a whole number $c^n$ will always be less than 1.

\subsection*{b}
Given that $c = 1$ and $n$ is a whole number $c^n$ will always equal 1. This is because when $\frac{1^{x}}{\infty}$ always equals 1.

\subsection*{c}
As long as $n$ is a even non-negative integer when calculating $\Theta(c^n)$ c will always be a positive. If $c = -5$ and $n = 2$ your result will be $\Theta(10)$ because the $n$ is even it brings the negative c to a positive. Same rule applies with a positive integer for c.

\end{document}
